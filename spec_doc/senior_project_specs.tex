\documentclass[11pt]{article}

\usepackage{fullpage}
\usepackage{amsfonts}
\usepackage{graphicx}
\usepackage{color}
\usepackage{listings}
%\usepackage[titletoc,title]{appendix}
\usepackage{float}

\setcounter{secnumdepth}{5}

\makeatletter
\renewcommand\paragraph{\@startsection{paragraph}{4}{\z@}%
                                     {-3.25ex\@plus -1ex \@minus -.2ex}%
                                     {0.0001pt \@plus .2ex}%
                                     {\normalfont\normalsize\bfseries}}
\renewcommand\subparagraph{\@startsection{subparagraph}{5}{\z@}%
                                     {-3.25ex\@plus -1ex \@minus -.2ex}%
                                     {0.0001pt \@plus .2ex}%
                                     {\normalfont\normalsize\bfseries}}
\makeatother

\begin{document}
\newcommand{\HRule}{\rule{\linewidth}{0.5mm}}

\begin{titlepage}
\begin{center}

~\\[4cm]

\textsc{\Large \textsc{Utah State University} }\\[0.5cm]

% Title
\HRule \\[0.4cm]
{ \huge \bfseries Passive Tracking Device \\ -- Senior Project -- \\ Specifications Document \\[0.4cm] }

\HRule \\[1.5cm]

% Author
\noindent
\begin{minipage}{0.4\textwidth}
\begin{flushleft} \large
\textsc{ Dallin Marshall }
\end{flushleft}
\end{minipage}%
\begin{minipage}{0.4\textwidth}
\begin{flushright} \large
\textsc{ Computer Engineering }
\end{flushright}
\end{minipage}

\vfill

% Bottom of the page
{\large \today}

\end{center}
\end{titlepage}

% \textcolor{white, black, red, green, blue, cyan, magenta, yellow} {text}
% \section*{text}
% \subsection*{text}
% \subsubsection*{text}
% \begin{enumerate/itemize}
% \begin{center} \includegraphics[scale=0.4]{path.png}\end{center}
% \lstinputlisting[language=Matlab, basicstyle=\footnotesize, firstline=40, lastline=81]{Matlab_Diary.txt}
%
%\begin{figure}
%   \centering
%       \includegraphics[scale=0.4]{Circuit_Schmatic.png}
%   \caption{Circuit Schematic}
%\end{figure}
%
%   \begin{figure}[H]
%       \centering
%           \begin{minipage}[b]{0.45\linewidth}
%               \includegraphics[scale=0.3]{./Fan_Data/Data2/AllFansA.png}
%               \caption{All three Fans 100\%}
%               \label{fig:minipage1}
%           \end{minipage}
%           \quad
%           \begin{minipage}[b]{0.45\linewidth}
%               \includegraphics[scale=0.3]{./Fan_Data/Data2/Fans1_2.png}
%               \caption{Fans 1 and 2 100\%}
%               \label{fig:minipage2}
%           \end{minipage}
%   \end{figure}


\thispagestyle{empty}
\tableofcontents
%\pagebreak
\vspace{2cm}
\pagebreak

%\thispagestyle{empty}
%\listoffigures
%%\listoftables
%\pagebreak

\pagenumbering{arabic}

\section{Scope}
This document outlines the specification of a passive tracking solution that could be used by enterprise level clients to track a 
large fleet of vehicles both ATVs or automobiles. This type of passive solution provides another alternative to companies wanting to track 
their fleet of vehicles that is both cheaper and less power intensive than other active tracking solutions.

\subsection{General}
This specification establishes the design, construction, performance, development, and test requirements for the passive tracking device, 
herein referred to as the PTD. Another focus of this project is to create a proof of concept that is compact enough to be contained within an ATV 
and use the ATVs battery to power the PTD. Many times it is difficult for companies to manage the location of their ATVs when they are used by 
a collective group of employees. This can make it difficult to know if the ATVs are really unaccounted for or merely being used by another 
employee.

\section{Applicable Documents}
The following documents of the exact issue shown shall form part of this specificaion to the extent specified herein. In the event of 
conflict between the requirements of this specification and any referenced document the order of precedence shall be 1. The contract, 
2. This specification, 3. Referenced documents.

\subsection{Government Documents}
The regulations in the following documents shall be met to allow the PTD to be legally used in most of North America (US and Canada).
\begin{itemize}
    \item The United States (US) FCC Part 15-2008.
    \item Canada's Industry Canada ICES-003:2004 Issue 4.
\end{itemize}

\section{Requirements}
This section outlines the requirements and specifications of the PTD. 

\subsection{Item Definitions}
The PTD will likely consist of 4 main subsystems, namely:
\begin{itemize}
    \item Controller Circuit -- The Controller Circuit shall act as an intermedary between the GPS and Cell Modules. I will be 
        responsible for directing the other modules to perform their functions at a specified time.
    \item Cell Network Interface -- The Cell Network Interface shall transmit the GPS data to the client via SMS messages.
    \item Global Positioning System -- The GPS shall calulate geographical location and transmit this data to Controller Circuit.
    \item Power Converter/Regulator -- The Power Converter/Regulator shall provide needed voltage and amperage from the ATV battery to 
        power the PTD.
\end{itemize}
The PTD system will be a passive system, meaning that the PTD will power down to a very low power state while not in use. At an appointed 
trigger it will power up and transmit it's location to the PTDs owner via a cell network. There is no plan for the owner of the PTD to be able to 
activate the device externally. The PTD shall opperate in this power-up, transmit location, power-down sequence.

\subsubsection{Interface Description/Functional Block Diagram}

%   \paragraph{Physical Interface Definitions}

%   \paragraph{Electrical Interface Definitions}

%   \paragraph{Functional Interface Definitions}

\subsection{Characteristics}
The following section provides a technical summary of the PTD's characteristics.

\subsubsection{Performance Characteristics}

\paragraph{Physical Characteristics} ~ \par 
The PTD shall meet the following physical requeirements:
    \subparagraph{Form Factor}  The PTD shall conform to the following form factor: 16cm by 8cm by 5cm 

\paragraph{Electrical Characteristics} ~ \par 
The PTD shall meet the following electrical requirements:
    \subparagraph{Power Constraints}  The PTD shall operate using the power provided by a standard ATV battery, specifically, 12V 11Ah battery.
    \subparagraph{Prototype Connections}  For the purposes of proof of concept the PTD may make use of a breadboard to connect the components.

\subsubsection{Environmental Characteristics}

\paragraph{Natural Characteristics}
The PTD shall meet the following natural evironmental characteristics. The PTD shall meet the requirements of this specification during and 
after exposure to any combination of any of the following natural environments. The PTD may be packaged to precluded exposure to any 
environments that would control the design.
    \subparagraph{Temperature Rating}  The PTD shall function between 0$^{\circ}$ C and 40$^{\circ}$ C

\paragraph{Induced Environment Characteristics}
The PTD shall meet the following induced evironmental characteristics. The PTD shall meet the requirements of this specification during 
and after exposure to any combination of any of the following induced environments. The PTD may be packaged to precluded exposure to any 
environments that would control the design.
    \subparagraph{Shock Test}  The PTD shall withstand mechanical shocks of 3 ft drop test 5 times onto concrete.
    \subparagraph{Vibration Test}  The PTD shall withstand vibrations of 1 ocilations per second with an amplitude of 1.5cm for 1 hour.
    \subparagraph{Dust Test}  The PTD shall function in a dusty environment, specifically, the shock and vibration tests shall be repeated 
after 15 grams of fine sand is applied to the device. 

\subsection{Electromagnetic Interference}
The PTD shall conform to the standards found in the \textit{Applicable Documents} section of this document.


\section{Verification}

\subsection{Item Definitions}

\subsubsection{Interface Description/Functional Block Diagram}

%   \paragraph{Physical Interface Definitions}

%   \paragraph{Electrical Interface Definitions}

%   \paragraph{Functional Interface Definitions}

\subsection{Characteristics}

\subsubsection{Performance Characteristics}

\paragraph{Physical Characteristics}
    \subparagraph{Form Factor}  The PTD shall be placed in an small box of dimention stated in requirements.

\paragraph{Electrical Characteristics}
    \subparagraph{Power Constraints}  The PTD shall function with only the power provied by an ATV battery. The PTD shall also be tested while the 
ATV's engine is actively running and under normal operation.
    \subparagraph{Prototype Connections}  The PTD shall have connections that function in any manner.

\subsubsection{Environmental Characteristics}

\paragraph{Natural Characteristics}
    \subparagraph{Temperature Rating}  The PTD shall be tested for full functionality at both ends of the temperature spectrum.

\paragraph{Induced Characteristics}
    \subparagraph{Shock Test}  The PTD shall function after the drop tests.
    \subparagraph{Vibration Test}  The PTD shall function after the viberation test is performed
    \subparagraph{Dust Test}  The PTD shall function after the dust test is performed.

\subsection{Electromagnetic Interference}
The EMI shall be measured and shall meet the requirements found in the \textit{Applicable Documents} section of this document.

\end{document}


